\subsection{Introduction}

\begin{frame}{LoRaWAN}
\framesubtitle{Introduction}
\begin{center}
\scalebox{0.8}{%
\begin{tikzpicture}[auto,node distance=1.2cm]
  \tikzstyle{comment}=[ right=2pt, font=\small, fill=white, text=black, draw=black, ]
  \tikzstyle{every state}=[rectangle,thick,draw=black,fill=gray!20,text=black, minimum width= 6cm, minimum height= 1.00cm ]
  \tikzstyle{smallstate}=[rectangle,thick,draw=black!80,fill=gray!10,text=black, minimum width= 6cm, minimum height= 0.25cm ]
  \tikzstyle{innerstate}=[rectangle,thick,draw=black,fill=gray!10,text=black, minimum width= 4cm, minimum height= 1.00cm ]

  \node[state,color=gray!40,fill=gray!20] at (0, 0) (A)            { Application Layer};
  \node[state,color=gray!40,fill=gray!20]         (E) [below of=A] { Network Layer};
  \node[state]         (F) [below of=E]                            { LoRaWAN };
  \node[state]         (G) [below of=F]                            { LoRa };

  \node[comment]       at (F.north west) {MAC Layer};
  \node[comment]       at (G.north west) {Physical Layer};
\end{tikzpicture}
}
\end{center}

\end{frame}

\subsection{Un protocole MAC ?}

\begin{frame}{LoRaWAN}
\framesubtitle{Un protocole MAC ?}

\begin{block}{MAC}
Elle sert d'interface entre la partie logicielle contrôlant la 
liaison d'un nœud (Contrôle de la liaison logique) et la couche 
physique (matérielle). Par conséquent, elle est différente selon 
le type de média physique utilisé (Ethernet, WLAN, …)
\end{block}

\end{frame}

\begin{frame}{LoRaWAN}
\framesubtitle{Un protocole MAC ?}
\begin{block}{}
{
  Quel moyen le plus simple d'interfacer la couche la couche physique ?
}
\end{block}
\end{frame}

\begin{frame}{LoRaWAN}
\framesubtitle{Aloha}
\end{frame}

\begin{frame}{LoRaWAN}
\framesubtitle{LoRaWAN n'est pas LoRa}
\end{frame}

\begin{frame}{LoRaWAN}
\framesubtitle{Organisation de LoRaWAN}
% Photo gateway + schema organisation
\end{frame}

\begin{frame}{LoRaWAN}
\framesubtitle{Pourquoi LoRa a autant de succes}
% Facilite de faire sa propre gateway
\end{frame}

\begin{frame}{LoRaWAN}
\framesubtitle{Aller plus loin}
\end{frame}
